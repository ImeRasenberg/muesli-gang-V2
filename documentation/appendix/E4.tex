\section{Code Exercise 4 (filled in)}}

\subsection{}\label{app:E4.1}

\begin{lstlisting}[style=Cstyle]
#include <stdio.h>
#include <time.h>
#include <assert.h>
#include <math.h>
#include "../downloads/mt19937.h"

#ifndef M_PI
#define M_PI 3.14159265358979323846
#endif

#define NDIM 3

/* Initialization variables */
const int mc_steps = 10000;
const int output_steps = 100;
const double packing_fraction = 0.6;
const double diameter = 1.0;
const double delta = 0.1;
const char* init_filename = "FCC_xyz.dat";

/* Simulation variables */
int N;
int n_particles = 0;
double radius;
double particle_volume;
double (*r)[3];
double *size;
double box[NDIM];


/* Functions */
void read_data(void){
	/*--------- Your code goes here -----------*/
	// degining the file
	FILE *read_cords;
	read_cords = fopen(init_filename, "r");
	
	// reading the first line to get the number of of particles that exist in the file (why is the exersise so weird???)
	fscanf(read_cords, "%i\n", &N); // the total number of particles
	// printf("%i\n", Loaded_Data.N);
	
	// making sure that the size will be correctly degined instead of having to asign it before hand
	// malloc is the memmory allocation commman which is wat we need to have exact size matrixes, only this satisfies me
	r = malloc(N * sizeof * r); // all the position vectors of all the particles
	size = malloc(N * sizeof * size); //The size of all particles
	
	// lets turn the above into a loop because i want to
	for(int i = 0; i<NDIM; i++){
		fscanf(read_cords, "%f %f", &box[i], &box[i]);
		
	}
	
	// now that we have arived at the paricles lets be happy
	for(int i = 0; i<N; i++){
		fscanf(read_cords, "%f %f %f %f", &r[i][0], &r[i][1], &r[i][2], &size[i]);
	}
	
	fclose(read_cords);
}
\end{lstlisting}

\subsection{}\label{app:E4.2}

\begin{lstlisting}[style=Cstyle]
	
	
\end{lstlisting}

\subsection{}\label{app:E4.3}

\begin{lstlisting}[style=Cstyle]
	
	
\end{lstlisting}


\subsection{}\label{app:E4.4}

\begin{lstlisting}[style=Cstyle]
	
	
\end{lstlisting}