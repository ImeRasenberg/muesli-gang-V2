\section{Exercise 5}

\subsection{}\label{app:E5.a}

\begin{lstlisting}[style=Cstyle]
	
	void change_volume(Loaded_Data l,float dV_m){
		/* given how we have defined the code before we should just be able to redefine the box sise, 
		scale all the particle possitions with new box length, and check for overlap
		
		*/
		float dV = (dsfmt_genrand()-0.5)*2*dV_m;
		float da = cbrt(dV);
		float r_c[l.N][3];
		
		
		for(int i=0; i<l.N; i++){
			
			for(int j=0; j<3; j++){
				
				float box_len = abs(l.box[j][0]-l.box[j][1]) ;
				float mult_fac = (box_len + da)/box_len ;
				r_c[i][j] = l.r[i][j] *mult_fac; 
				
			}
		}
		
		for(int i; i<3;i++){
			l.box[i][0]+= dV;
			l.box[i][0]-= dV;
		}
		
	}
	
	int check_overlap_volume(Loaded_Data l, float **r){
		// we want to compare each particle to each other particle so 2 loops over particle index
		for(int i=0; i<l.N; i++){
			// chosing the particle
			float *p = r[i];
			for(int j=0; j<l.N; j++){
				float *p_c = r[j];
				
				float dist = 0;
				
				for(int k=0;k<3;k++){
					dist += (p[k] + p_c[k])*(p[k] + p_c[k]);
					
				}
				float sum_raduses = 0.5 * (l.size[i] + l.size[j]);
				
				
				if (dist < sum_raduses*sum_raduses) {
					return 1; // Overlap detected
				}
			}
		}
		return 0;
	}
	
\end{lstlisting}