\section{Monte Carlo simulation of Hard spheres in the NVT ensemble}

In this exercise we have a couple of assignment that have to be done. The web app used for plotting is found at \url{https://webspace.science.uu.nl/~herme107/viscol/}.




\subsection{}

Here we had to make code that tiled the space whit spheres in a cubic lattice formation. The code for the generation of this lattice is found in Appendix~\ref{app:E4.a}.

On the web app for plotting this lattice, Figure~\ref{fig: cubic lattice} was made using the file that was generated.

\begin{figure}[H]
	\centering
	\includegraphics[width=0.5\linewidth]{graphs/E4/Cubic.png}
	\caption{Here the cubic lattice generated by the code is graphed}
	\label{fig: cubic lattice}
\end{figure}




\subsection{} \label{sec:4b}
We want to know the maximum packing density for spheres in a cubic lattice.

To get this first we need the lattice vector equation 
\begin{equation}
	\vec{R} = N_1 \vec{a_1}+N_2 \vec{a_2}+N_3 \vec{a_3}.
	\label{eq: lattice vector}
\end{equation}
Here the $N_i \in \mathcal{Z}$ is the counting number and $\vec{a_i}$ are the primitive translation vectors.

For the cubic case the vectors are unit vectors times the radius of the atoms. Dividing this up into unit cells gives us that only one atom may exist in the unit cell. meaning that the occupied fraction

\begin{eqnarray}
	f_{o} = \frac{V_p}{V_{uc}} = \frac{\frac{4}{3}\pi (\frac{a}{2})^3}{a^3} = \frac{\pi}{6}
\end{eqnarray}

Here $V_p$ is the volume of particles occupying the unit cell, $V_{uc}$ is the volume of the unit cell and $a$ is the diameter of the particle. 




\subsection{}
Here we had to make code that tiled a space with spheres in a face-centered cubic (FCC) lattice. The code for the generation of this lattice is found in Appendix~\ref{app:E4.b}.

On the web app for plotting this lattice, Figure~\ref{fig: Fcc lattice} was made using the file that was generated.

\begin{figure}[H]
	\centering
	\includegraphics[width=0.5\linewidth]{graphs/E4/FCC.png}
	\caption{Here the cubic lattice generated by the code is graphed}
	\label{fig: Fcc lattice}
\end{figure}




\subsection{}
We want to know the maximum packing density for spheres in a FCC lattice.

to get this we will use the same process as in \ref{sec:4b}. We know that our primitive translation vectors are
\begin{equation*}
	a_1 = \frac{a}{\sqrt{2}}\begin{pmatrix} 0 \\ 1 \\ 1 \end{pmatrix},\ a_2 = \frac{a}{\sqrt{2}}\begin{pmatrix} 1 \\ 0 \\ 1 \end{pmatrix},\ a_3 = \frac{a}{\sqrt{2}}\begin{pmatrix} 1 \\ 1 \\ 0 \end{pmatrix}
\end{equation*}
If we do this we have a problem, choosing the unit cell with length $a$ in each direction leads to a non-translation symmetric unit cell. To get a translation symmetric unit cell we have to chose the length of the unit cell to be $\sqrt{2}a$. 

Looking at the unit cell we can see that for each corner $1/8$th of a sphere exists and we get $1/2$ for each face of the cube, meaning the cube contains $8*1/8 + 6*1/2 = 4$ spheres

From this we know that if we look at how much particles would be in a unit cell we see that this would be 

\begin{eqnarray}
	f_o = \frac{4V_p}{V_{uc}} = \frac{\frac{16}{3}\pi (\frac{a}{2})^3}{(\sqrt{2}a)^3} = \frac{\pi}{3\sqrt{2}}
\end{eqnarray}




\subsection{}






