\section{Monte Carlo simulation of Hard spheres in the NVT ensemble (correct)}



\subsection{}

Here we had to make code that tiled the space whit spheres in a cubic lattice formation. The code for the generation of this lattice is found in Appendix~\ref{app:E4.a}.

On the web app for plotting this lattice, Figure~\ref{fig: cubic lattice} was made using the file that was generated.

\begin{figure}[H]
	\centering
	\includegraphics[width=0.5\linewidth]{graphs/E4/Cubic.png}
	\caption{Here the cubic lattice generated by the code is graphed}
	\label{fig: cubic lattice}
\end{figure}




\subsection{} \label{sec:4b}
We want to know the maximum packing density for spheres in a cubic lattice.

To get this first we need the lattice vector equation 
\begin{equation}
	\vec{R} = N_1 \vec{a_1}+N_2 \vec{a_2}+N_3 \vec{a_3}.
	\label{eq: lattice vector}
\end{equation}
Here the $N_i \in \mathcal{Z}$ is the counting number and $\vec{a_i}$ are the primitive translation vectors.

For the cubic case the vectors are unit vectors times the radius of the atoms. Dividing this up into unit cells gives us that only one atom may exist in the unit cell. meaning that the occupied fraction

\begin{eqnarray}
	f_{o} = \frac{V_p}{V_{uc}} = \frac{\frac{4}{3}\pi (\frac{a}{2})^3}{a^3} = \frac{\pi}{6}
\end{eqnarray}

Here $V_p$ is the volume of particles occupying the unit cell, $V_{uc}$ is the volume of the unit cell and $a$ is the diameter of the particle. 




\subsection{}
Here we had to make code that tiled a space with spheres in a face-centered cubic (FCC) lattice. The code for the generation of this lattice is found in Appendix~\ref{app:E4.b}.

On the web app for plotting this lattice, Figure~\ref{fig: Fcc lattice} was made using the file that was generated.

\begin{figure}[H]
	\centering
	\includegraphics[width=0.5\linewidth]{graphs/E4/FCC.png}
	\caption{Here the cubic lattice generated by the code is graphed}
	\label{fig: Fcc lattice}
\end{figure}




\subsection{}
We want to know the maximum packing density for spheres in a FCC lattice.

to get this we will use the same process as in \ref{sec:4b}. We know that our primitive translation vectors are
\begin{equation*}
	a_1 = \frac{a}{\sqrt{2}}\begin{pmatrix} 0 \\ 1 \\ 1 \end{pmatrix},\ a_2 = \frac{a}{\sqrt{2}}\begin{pmatrix} 1 \\ 0 \\ 1 \end{pmatrix},\ a_3 = \frac{a}{\sqrt{2}}\begin{pmatrix} 1 \\ 1 \\ 0 \end{pmatrix}
\end{equation*}
If we do this we have a problem, choosing the unit cell with length $a$ in each direction leads to a non-translation symmetric unit cell. To get a translation symmetric unit cell we have to chose the length of the unit cell to be $\sqrt{2}a$. 

Looking at the unit cell we can see that for each corner $1/8$th of a sphere exists and we get $1/2$ for each face of the cube, meaning the cube contains $8*1/8 + 6*1/2 = 4$ spheres

From this we know that if we look at how much particles would be in a unit cell we see that this would be 

\begin{eqnarray}
	f_o = \frac{4V_p}{V_{uc}} = \frac{\frac{16}{3}\pi (\frac{a}{2})^3}{(\sqrt{2}a)^3} = \frac{\pi}{3\sqrt{2}}
\end{eqnarray}




\subsection{}
The code for this \textbf{$read\_data()$} subroutine is found in Appendix~\ref{app:4.2}.

we first initialize the file that we are interested in, we then open the file and scan the first line. The first line contains the code number of particles in the system using which we know how much of the file we need to read. Then we have a line of code to read out the 3 box dimensions that are defined. After this we know the file contains (x,y,z,r) where the x,y,z are the coordinates and the r is the radius of the box. We read these putting x,y,z into the "r" vector (1D pointer) and r into the "size" vector (1D pointer).


\subsection{}
The code for this \textbf{$move\_particle()$} subroutine is found in Appendix~\ref{app:E4.4}.

First we generate a random particle index for our position pointer. 

Then a random amplitude for our translation is generated in the domain [-delta,delta]. After a random translation direction in 3D is generated, which is then normalised and scaled by the amplitude of the translation.

This translation is validated using the move $check\_particle\_overlap()$ if it is found to not overlap the translation is executed. 

After the translation we look if the particle is still in the box. If it is not the periodic boundaries are imposed.



\subsection{}
The code for this \textbf{$check\_particle\_overlap()$} subroutine is found in Appendix~\ref{app:E4.e}.

It just tests that the random particle can do the translation without overlapping. This is done by checking its distance to all other particles and seeing if their radius's added are smaller then their distance.




\subsection{}

The results of this NVT ensemble evolution of a cubic lattice are plotted in Figure~\ref{fig: cubic time evolved}.

\begin{figure}[H]
	\centering
	\includegraphics[width=0.5\linewidth]{graphs/E4/NTV procession.png}
	\caption{Here the cubic lattice generated is perturbed 100000 times to give this configuration}
	\label{fig: cubic time evolved}
\end{figure}



\subsection{}

Figure~\ref{fig:packing density sweep NVT} shows the NVT simulation at different packing densities.

At a density .55 the particles still seem to have some underlying structure, while at a density of .4 it doesn't seem to have any underlying structure, with .5 and .45 its hard to tell, so somewhere in between the structure disappears meaning it melts in the range [.4 - .55].

\begin{figure}[H]
	\centering
		\subfloat[the packing density is 0.40]{\includegraphics[width=0.35\linewidth]{graphs/E4/40.png}}
	\subfloat[the packing density is 0.45]{\includegraphics[width=0.35\linewidth]{graphs/E4/45.png}}\\
	\subfloat[the packing density is 0.50]{\includegraphics[width=0.35\linewidth]{graphs/E4/50.png}}
	\subfloat[the packing density is 0.55]{\includegraphics[width=0.35\linewidth]{graphs/E4/55.png}}
	\caption{The packing density changed}
	\label{fig:packing density sweep NVT}
\end{figure}




















